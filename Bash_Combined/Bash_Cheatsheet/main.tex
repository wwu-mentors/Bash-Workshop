%%%%%%%%%%%%%%%%%%%%%%%%%%%%%%%%%%%%%%%%%
% Cheatsheet
% LaTeX Template
% Version 1.0 (12/12/15)
%
% This template has been downloaded from:
% http://www.LaTeXTemplates.com
%
% Original author:
% Michael Müller (https://github.com/cmichi/latex-template-collection) with
% extensive modifications by Vel (vel@LaTeXTemplates.com)
%
% License:
% The MIT License (see included LICENSE file)
%
%%%%%%%%%%%%%%%%%%%%%%%%%%%%%%%%%%%%%%%%%
%
% Modified by Ted Weber
%----------------------------------------------------------------------------------------
%   PACKAGES AND OTHER DOCUMENT CONFIGURATIONS
%----------------------------------------------------------------------------------------
\documentclass[11pt]{scrartcl} % 11pt font size
\usepackage[utf8]{inputenc} % Required for inputting international characters
\usepackage[T1]{fontenc} % Output font encoding for international characters
\usepackage[margin=0pt, landscape]{geometry} % Page margins and orientation
\usepackage{graphicx} % Required for including images
\usepackage{color} % Required for color customization
\definecolor{mygray}{gray}{.75} % Custom color
\usepackage{url} % Required for the \url command to easily display URLs
\usepackage[ % This block contains information used to annotate the PDF
colorlinks=false, 
pdftitle={Cheatsheet}, 
pdfauthor={John Smith}, 
pdfsubject={Compilation of useful shortcuts}, 
pdfkeywords={Random Software, Cheatsheet}
]{hyperref}
\setlength{\unitlength}{1mm} % Set the length that numerical units are measured in
\setlength{\parindent}{0pt} % Stop paragraph indentation
\renewcommand{\dots}{\ \dotfill{}\ } % Fills in the right amount of dots
\newcommand{\command}[2]{\textbf{#1}~\dotfill{}~#2\\} % Custom command for adding a shortcut
\newcommand{\sectiontitle}[1]{\large{\textbf{#1}} \ \\} % Custom command for subsection titles
%----------------------------------------------------------------------------------------
\begin{document}
\begin{picture}(297,210) % Create a container for the page content
%----------------------------------------------------------------------------------------
%   TITLE SECTION 
%----------------------------------------------------------------------------------------
\put(10,200){ % Position on the page to put the title
\begin{minipage}[t]{210mm} % The size and alignment of the title
\section*{Bash Cheatsheet} % Title
\end{minipage}
}
%----------------------------------------------------------------------------------------
%   FIRST COLUMN SPECIFICATION
%----------------------------------------------------------------------------------------
\put(10,180){ % Divide the page
\begin{minipage}[t]{85mm} % Create a box to house text

%----------------------------------------------------------------------------------------
%   HEADING ONE
%----------------------------------------------------------------------------------------
\sectiontitle{Directory Commands}
           
\command{cd}{Change current working directory.}
\command{cd -}{Return to prev working directory.}
\command{..}{Represents parent of current directory.}
\command{.}{Represents current working directory.}
\command{pwd}{Print name of current/working directory.}
\command{mkdir}{Make a new directory.}
\command{rmdir}{Remove empty directory.}
\command{rm}{Remove file or directory.}
%----------------------------------------------------------------------------------------
%   HEADING TWO
%----------------------------------------------------------------------------------------               
            
\sectiontitle{File Commands}

\command{\textbf{ls}}{List information about files.}
\command{ls -la}{List detailed information about all files.}
\command{touch}{Change file timestamps.}
\command{mv}{Move a file OR rename a file.}
\command{cat}{Concatenate files and print to stdout.}
\command{more}{View file one page at a time.}
\command{less}{Enhanced version of \textbf{more} command.}
\command{which}{Find where program is located.}
\command{whereis}{Like \textbf{which}, but broader results.}
\command{whatis}{Get short description of a program.}
\command{xdg-open}{Open file with default program.}
%----------------------------------------------------------------------------------------
%   HEADING THREE
%----------------------------------------------------------------------------------------   

\sectiontitle{System Commands}

\command{\textasciicircum D}{End process (more graceful than \textasciicircum C).}
\command{\textasciicircum C}{Kill the current process.}
\command{top}{Display system processes.}
\command{[program] \textgreater \textbf{ [file]}}{Redirect program output to file (overwrites).}
\command{[program] \textgreater\textgreater [file]}{Append output to file.}
\command{[program1] \textbar\textbf{ [program2]}}{Make output from program1 input to program2}


\end{minipage} % End the first column of text
} % End the first division of the page
%----------------------------------------------------------------------------------------
%   SECOND COLUMN SPECIFICATION 
%----------------------------------------------------------------------------------------
\put(105,180){ % Divide the page
\begin{minipage}[t]{85mm} % Create a box to house text
%----------------------------------------------------------------------------------------
%   HEADING FOUR
%----------------------------------------------------------------------------------------

\sectiontitle{Navigation}

\command{<tab>}{Auto-completes user input.}
\command{<up/down arrow>}{Cycle through command history.}
\command{\textasciicircum R}{Reverse search through command history.}
\command{clear}{Clears terminal view (also \textasciicircum l)}
\command{apropos}{Search for commands whose description contains a certain term.}
\command{!!}{Run immediately previous command.}
\command{![str]}{Run last command starting with str.}
\command{[cmd1] ; [cmd2]}{Run cmd1, then cmd2.}
\command{man}{Access the system manual pages.}

%----------------------------------------------------------------------------------------
%   HEADING FIVE
%----------------------------------------------------------------------------------------
\sectiontitle{Misc Commands}

\command{echo}{Displays a line of text}
\command{history}{Lists all previous commands}


%----------------------------------------------------------------------------------------
%   HEADING SIX
%----------------------------------------------------------------------------------------
\sectiontitle{Legend}

\command{\textasciicircum}{Abbreviation for Ctrl}
\command{\textless keyboard\_ button\textgreater}{Denotes button on keyboard}
\command{[user\_ input]}{Denotes user input of a given kind}
\command{cwd}{Stands for Current Working Directory}
%Use framebox to hold legend?

%----------------------------------------------------------------------------------------
\end{minipage} % End the second column of text
} % End the second division of the page
%----------------------------------------------------------------------------------------
%----------------------------------------------------------------------------------------
%   THIRD COLUMN SPECIFICATION 
%----------------------------------------------------------------------------------------
\put(200,180){ % Divide the page
\begin{minipage}[t]{85mm} % Create a box to house tex
%----------------------------------------------------------------------------------------
%   IMPORTANT FILES
%----------------------------------------------------------------------------------------
\sectiontitle{Example .bashrc files}
\begin{itemize}
    \item \url{https://crunchbang.org/forums/viewtopic.php?id=1093}
    \item \url{http://tldp.org/LDP/abs/html/sample-bashrc.html}
\end{itemize}

\vspace{\baselineskip} % Whitespace before the next section

%----------------------------------------------------------------------------------------
%   LINKS AND INFORMATION
%----------------------------------------------------------------------------------------
\sectiontitle{Other Links and Information}
\textbf{\textit{Bash Scripting:}} 
\begin{itemize}
    \item \url{http://ryanstutorials.net/bash-scripting-tutorial/bash-script.php}
    \item \url{http://tldp.org/HOWTO/Bash-Prog-Intro-HOWTO.html}
\end{itemize}

\textbf{\textit{Shells:}} 
\begin{itemize}
    \item \url{https://en.wikipedia.org/wiki/Shell_(computing)}
    \item \url{https://www.ibm.com/developerworks/library/l-linux-shells/}
\end{itemize}


%----------------------------------------------------------------------------------------
%   FOOTNOTE
%----------------------------------------------------------------------------------------
\vspace{\baselineskip}
\linethickness{0.5mm} % Thickness of the footer line
{\color{mygray}\line(1,0){30}} % Print the line with a custom color

\footnotesize{
Created by John Smith, 2015\\ 
Modified by S. Bowen, S. Gunderson, T. Weber, 2017\\
\url{http://johnsmith.com/}\\
                
Released under the MIT license.
}
%----------------------------------------------------------------------------------------
\end{minipage} % End the third column of text
} % End the third division of the page
\end{picture} % End the container for the entire page
%----------------------------------------------------------------------------------------
\end{document}